%%%%%%%%%%%%%%%%%%%%%%%%%%%%%%%%%%%%%%%%%
% Cleese Assignment (For Teachers)
% LaTeX Template
% Version 2.0 (27/5/2018)
%
% This template originates from:
% http://www.LaTeXTemplates.com
%
% Author:
% Vel (vel@LaTeXTemplates.com)
%
% License:
% CC BY-NC-SA 3.0 (http://creativecommons.org/licenses/by-nc-sa/3.0/)
% 
%%%%%%%%%%%%%%%%%%%%%%%%%%%%%%%%%%%%%%%%%

%----------------------------------------------------------------------------------------
%	PACKAGES AND OTHER DOCUMENT CONFIGURATIONS
%----------------------------------------------------------------------------------------

\documentclass[11pt]{scrartcl}
\usepackage{xcolor}

\definecolor{color1}{rgb}{0.22,0.45,0.70}

%%%%%%%%%%%%%%%%%%%%%%%%%%%%%%%%%%%%%%%%%
% Wenneker Assignment
% Structure Specification File
% Version 2.0 (12/1/2019)
%
% This template originates from:
% http://www.LaTeXTemplates.com
%
% Authors:
% Vel (vel@LaTeXTemplates.com)
% Frits Wenneker
%
% License:
% CC BY-NC-SA 3.0 (http://creativecommons.org/licenses/by-nc-sa/3.0/)
% 
%%%%%%%%%%%%%%%%%%%%%%%%%%%%%%%%%%%%%%%%%

%----------------------------------------------------------------------------------------
%	PACKAGES AND OTHER DOCUMENT CONFIGURATIONS
%----------------------------------------------------------------------------------------

\usepackage{amsmath, amsfonts, amsthm} % Math packages

\usepackage{listings} % Code listings, with syntax highlighting

%\usepackage[english]{babel} % English language hyphenation

\usepackage{graphicx} % Required for inserting images
\graphicspath{{Figures/}{./}} % Specifies where to look for included images (trailing slash required)

\usepackage{booktabs} % Required for better horizontal rules in tables

\numberwithin{equation}{section} % Number equations within sections (i.e. 1.1, 1.2, 2.1, 2.2 instead of 1, 2, 3, 4)
\numberwithin{figure}{section} % Number figures within sections (i.e. 1.1, 1.2, 2.1, 2.2 instead of 1, 2, 3, 4)
\numberwithin{table}{section} % Number tables within sections (i.e. 1.1, 1.2, 2.1, 2.2 instead of 1, 2, 3, 4)

\setlength\parindent{0pt} % Removes all indentation from paragraphs

\usepackage{enumitem} % Required for list customisation
\setlist{noitemsep} % No spacing between list items

%----------------------------------------------------------------------------------------
%	DOCUMENT MARGINS
%----------------------------------------------------------------------------------------

\usepackage{geometry} % Required for adjusting page dimensions and margins

\geometry{
	paper=a4paper, % Paper size, change to letterpaper for US letter size
	top=2.5cm, % Top margin
	bottom=3cm, % Bottom margin
	left=3cm, % Left margin
	right=3cm, % Right margin
	headheight=0.75cm, % Header height
	footskip=1.5cm, % Space from the bottom margin to the baseline of the footer
	headsep=0.75cm, % Space from the top margin to the baseline of the header
	%showframe, % Uncomment to show how the type block is set on the page
}

%----------------------------------------------------------------------------------------
%	FONTS
%----------------------------------------------------------------------------------------

%\usepackage[utf8]{inputenc} % Required for inputting international characters
%\usepackage[T1]{fontenc} % Use 8-bit encoding
%
%\usepackage{fourier} % Use the Adobe Utopia font for the document

\usepackage[utf8x]{inputenc} % Required for inputting international characters
%\usepackage[T1]{fontenc} % Output font encoding for international characters
\usepackage{xltxtra}
\setmainfont{EHUSerif}
\setsansfont{EHUSans}

%----------------------------------------------------------------------------------------
%	SECTION TITLES
%----------------------------------------------------------------------------------------

\usepackage{sectsty} % Allows customising section commands

\sectionfont{\vspace{6pt}\centering\normalfont\scshape} % \section{} styling
\subsectionfont{\normalfont\bfseries} % \subsection{} styling
\subsubsectionfont{\normalfont\itshape} % \subsubsection{} styling
\paragraphfont{\normalfont\scshape} % \paragraph{} styling

%----------------------------------------------------------------------------------------
%	HEADERS AND FOOTERS
%----------------------------------------------------------------------------------------

\usepackage{scrlayer-scrpage} % Required for customising headers and footers

\ohead*{} % Right header
\ihead*{} % Left header
\chead*{} % Centre header

\ofoot*{} % Right footer
\ifoot*{} % Left footer
\cfoot*{\pagemark} % Centre footer
 % Include the file specifying the document structure and custom commands
\usepackage[basque]{babel}

\title{	
	\normalfont\normalsize
	\textsc{Kimika Fakultatea \\ UPV/EHU }\\ % Your university, school and/or department name(s)
	\vspace{25pt} % Whitespace
	\color{color1}{{\rule{\linewidth}{0.5pt}}}\\ % Thin top horizontal rule
	\vspace{20pt} % Whitespace
	{\huge Kimika Fisikoa II. Praktikak I.}\\ % The assignment title
	\vspace{12pt} % Whitespace
	\rule{\linewidth}{2pt}\\ % Thick bottom horizontal rule
	\vspace{12pt} % Whitespace
}

\author{\LARGE Ion Mitxelena, David de Sancho,Txema Mercero, \\
Xabier Lopez } % Your name

\date{\normalsize\today}

%----------------------------------------------------------------------------------------

\begin{document}
\maketitle
%----------------------------------------------------------------------------------------

%\assignmentSection{Errotazio Espektroskopia}

%----------------------------------------------------------------------------------------
%	QUESTION 1
%----------------------------------------------------------------------------------------

\section{Partikula bat Kaxa batean}

%\begin{question} 
	\subsection{Energia.}
         \begin{itemize}
		 \item Kalkula ezazu n=1 -> n=2 trantsizioaren energia. Nola aldatzen da energ�a hau L-rekin? 
			 Irudika ezazu. Ze ondorio atera ditzakezu?
           \item Azter itzazu (5,3,3), (3,5,3), (3,3,5) egoeren energiak, L=10 a.u.-tako kubo batean gaudenean. 
		   Energiak kalkulatu 
         baino lehen, zeozer aurresan zenezake? Zein dira egoera hauen energiak (a.u.-etan)
	\item Kubo bat beharrean, L$_{x}$=L$_{y}$$\neq$L$_{z}$ kaxa batean, zein izango dira aurreko egoeren energiak, L$_{x}$=10 eta L$_{z}$=12
izanda? Konparatu energia aurreko kasuarekin.
	%$\psi_{1}$, eta $\psi_{1,1,1}$ egoerak eta eztabaidatu.
        \end{itemize}

	\subsection{Irudikatu honako egoerak:}
	\begin{itemize}
		\item $ \psi(3), \psi(50) \quad eta \quad \psi(3)^{2} \quad $ Zer ikusi daiteke kasu bakoitzean?
	       \item Irudikatu  $\psi(2,1,1)$ egoera. Zer gogoraerazten dizu emaitzak? Eztabaidatu.  		
		\item  Zein da aurreko egoera bakoitzaren energia, L=10 a.u. aldeko kubo batean bagaude? 
	\end{itemize}
%\end{question}

	\newpage
	\section{Hidrogeno Atomoa}
%\begin{question}
	\subsection{Energia Potentziala}
	\begin{itemize}
           \item Zein da energia potentziala r=2.4 a.u. denean?
           \item Irudikatu ezazu V(r) r= \{0..8\ \} balio tartearentzat, eta energia \ \{-5,0 \} tartean, unitate atomikoetan. Eztabaidatu grafika.
	\end{itemize}
	\subsection{Irudikatu d$_{xz}$ orbitala 2D eta 3D-tan.}
	\begin{itemize}
		\item Eztabaidatu ikusten duzuna n=3 eta n=5 denean.
		\item Zein dira Orbital hauei dagozkien zenbaki kuantiko guztiak?
	\end{itemize}
	\subsection{Egin ditugun hurbilketekin, zer gertatuko litzateke Deuterio atomo bat
izango bagenu? Zein izango litzateke bere energia?}
\subsection{Zein izango da He$^{+}$ katioiaren energia? Kalkula genezake egindako hurbilketekin?}


%\end{question}


	%\subsection{Irudikatu d$_{xz}$ orbitala 2D eta 3D eta eztabaidatu ikusten duzunam n=3 eta n=5 denean.}


\end{document}
