%Cambios de color
\include{tikzSetup}
%Paleta
\usepackage{xcolor}

\definecolor{azulfct}{RGB}{0,42,96}
\definecolor{gris}{RGB}{90,95,97}
\definecolor{azul}{RGB}{0,112,255}
\definecolor{naranja}{RGB}{255,143,0}
\definecolor{blanco}{RGB}{255,255,255}

\definecolor{myblue}{rgb}{0.086, 0.3098, 0.5255} % 154F86
\definecolor{mybluedd}{HTML}{2b8cbe} %Dark HTML
\definecolor{myblueddd}{HTML}{173753} %Dark HTML 173753
\definecolor{mybluedddd}{HTML}{1B4353} %DarkDark HTML 1B4353
\definecolor{myblueF}{HTML}{23516d} % 154F86
\definecolor{myblue5}{rgb}{0.08,0.58,0.89}
\definecolor{mybluel}{HTML}{3E75a9} %Light  6DAEDB
\definecolor{mybluell}{HTML}{2d5d8a} %LightLight


%del fichero beamercolorthemebeaver.sty
\setbeamercolor{structure}{fg=azulfct}
% \setbeamercolor{structure}{fg=azul} %Cambiado

\setbeamercolor{palette primary}{fg=azulfct,bg=gris!70}
\setbeamercolor{palette secondary}{fg=azulfct,bg=gris!80}
\setbeamercolor{palette tertiary}{fg=azulfct,bg=gris!90}
\setbeamercolor{palette quaternary}{fg=azulfct,bg=gris}

% \setbeamercolor{titlelike}{parent=palette quaternary}
% 
% \setbeamercolor{block title}{fg=azulfct,bg=gris}
% \setbeamercolor{block title alerted}{use=alerted text,fg=azulfct,bg=alerted text.fg!75!bg}
% \setbeamercolor{block title example}{use=example text,fg=azulfct,bg=example text.fg!75!bg}
% 
% \setbeamercolor{block body}{parent=normal text,use=block title,bg=block title.bg!25!bg}
% \setbeamercolor{block body alerted}{parent=normal text,use=block title alerted,bg=block title alerted.bg!25!bg}
% \setbeamercolor{block body example}{parent=normal text,use=block title example,bg=block title example.bg!25!bg}
% 
% \setbeamercolor{sidebar}{bg=gris!70}
% 
% \setbeamercolor{palette sidebar primary}{fg=azulfct}
% \setbeamercolor{palette sidebar secondary}{fg=azulfct!75}
% \setbeamercolor{palette sidebar tertiary}{fg=azulfct!75}
% \setbeamercolor{palette sidebar quaternary}{fg=azulfct}

% \setbeamercolor*{separation line}{}
% \setbeamercolor*{fine separation line}{}


%%Mis cambios

\setbeamercolor{frametitle}{fg=azulfct,bg=gris}
\setbeamercolor{normal text}{fg=azulfct}


% \usepackage[utf8]{inputenc}  %parece que no hace falta con xelatex
% \usepackage{default}

%La plantilla
% \usetheme{Goettingen}

%Para que los itemize sean circulos
\useinnertheme{circles}


\usepackage[no-math]{fontspec}
\setsansfont{EHUSans}
%Para algún texto particular
\newfontfamily\myEHUSerif{EHUSerif}     %Uso, entre paréntesis si es necesario   \my_EHUSerif
 


% \usepackage[many]{tcolorbox}
% \makeatletter
% \setbeamertemplate{frametitle}{%
%   \nointerlineskip%
%   \usebeamerfont{headline}%
%   \nointerlineskip%
%   \hbox{\hspace{-0.09\paperwidth}%
%   \begin{tcolorbox}[
%     enhanced,
%     boxrule=0pt,   %Elimino un caja de 15pt que ponía a la izquierda
%     colframe=naranja,
%     arc=0pt,
%     outer arc=0pt,
%     colback=green,
%     colupper=naranja,
%     width=\paperwidth+2mm,
%     toprule=0pt,
%     bottomrule=0pt,
%     rightrule=0pt,
%     left=15pt,
%   ]%
% %    {\usebeamercolor{frametitle}\usebeamerfont*{frametitle}\strut\insertframetitle\strut} Quito dos strut que meten un espacio y hacen el frametitle más ancho
%     {\usebeamercolor{frametitle}\usebeamerfont*{frametitle}\insertframetitle}
%     \ifx\insertframesubtitle\@empty%
%       \strut\par%
%     \else%
%      \par{\usebeamerfont*{framesubtitle}{\usebeamercolor[fg]{framesubtitle}\insertframesubtitle}\par}%
%     \fi
%   \end{tcolorbox}%
%   }
%   \hrule width \paperwidth heigh 100pt
% 
%   \nointerlineskip
% }
% \makeatother


\setbeamertemplate{frametitle}{%
    \usebeamerfont{frametitle}\vspace{.2\baselineskip}\insertframetitle%
    \vphantom{g}% To avoid fluctuations per frame
    %\hrule% Uncomment to see desired effect, without a full-width hrule
    \par% <-- added
    
    \ifx\insertframesubtitle\@empty%
      \strut\par%
    \else%
     \par{\usebeamerfont*{framesubtitle}{\usebeamercolor[fg]{framesubtitle}\vspace{.3\baselineskip}\insertframesubtitle}\par}%
    \fi     
    \hspace*{-\dimexpr0.5\paperwidth-0.5\textwidth}% <-- calculation of left margin width
    %a paperwidth=128mm=364.16pt   1mm=2.845pt 
%Para 4/3    
%     \mbox{\rule[.4\baselineskip]{224.59pt}{1pt}\hspace{1pt}%  % para que sea una continuación de linea, sino introduce `` ''
% \rule[.4\baselineskip]{85.17pt}{1pt}\hspace{1pt}%
% \rule[.4\baselineskip]{31.91pt}{1pt}\hspace{1pt}%
% \rule[.4\baselineskip]{11.57pt}{1pt}\hspace{1pt}%
% \rule[.4\baselineskip]{3.80pt}{1pt}\hspace{1pt}%
% \rule[.4\baselineskip]{1pt}{1pt}
%Para 16/9    
    \mbox{\rule[.4\baselineskip]{279.82pt}{1pt}\hspace{1pt}%  % para que sea una continuación de linea, sino introduce `` ''
\rule[.4\baselineskip]{105.46pt}{1pt}\hspace{1pt}%
\rule[.4\baselineskip]{40.05pt}{1pt}\hspace{1pt}%
\rule[.4\baselineskip]{14.68pt}{1pt}\hspace{1pt}%
\rule[.4\baselineskip]{5.99pt}{1pt}\hspace{1pt}%
\rule[.4\baselineskip]{2.29pt}{1pt}\hspace{1pt}%
\rule[.4\baselineskip]{1pt}{1pt}
}
\vspace{-.5em}  % subo un poco porqu si no el texto principal empiza muy abajo
}



%Para las imagenes
\usepackage{graphicx}
\graphicspath{{Images/}}
%Para fijar imagenes
\usepackage[absolute,overlay]{textpos}   % añadir ,showboxes para ver la caja de texto, no tiene borde por abajo, es donde empieza el texto y de ahí para abajo
\setlength{\TPHorizModule}{128mm}   % 1 unidad de ancho es textpos es el ancho de la diapositiva
\setlength{\TPVertModule}{96mm}     % 1 unidad de alto es textpos es el ancho de la diapositiva


%Kentzen ditugu beheko nabigatzeko botoiak
\setbeamertemplate{navigation symbols}{}

%Page number esto pone 10/96
% \setbeamertemplate{footline}[frame number]
% Esto quita el numero total de frames
\setbeamertemplate{footline}{% 
  \hfill% 
  \usebeamercolor[fg]{page number in head/foot}% 
  \usebeamerfont{page number in head/foot}% 
  \insertframenumber%
  %\,/\,\inserttotalframenumber
  \kern1em\vskip2pt% 
}


%Multirow
\usepackage{multirow}


%Para las citas de la esquina
% \usepackage[absolute]{textpos}
% \setlength{\TPHorizModule}{10mm}
% \setlength{\TPVertModule}{\TPHorizModule}
\usepackage{tikz}
\usepackage[]{media9}


%%% Time in presentation

\usepackage[font=Times,timeinterval=1, timeduration=2.0, timedeath=0, fillcolorwarningsecond=white!60!yellow,
timewarningfirst=50,timewarningsecond=80,resetatpages=all]{tdclock}

\usepackage{cancel}
\usepackage{amsmath}
\usepackage{listings}
\usepackage{animate}
\usepackage{setspace}
\usepackage{changepage}
\usepackage{caption}
\usepackage{textpos}
\usepackage{fancybox}
\usepackage{pstricks}
\usepackage{media9}
\usepackage{tabu}
\hypersetup{pdfpagemode=UseNone}

\usepackage{colortbl}
\usepackage{overpic}
\usepackage{geometry}
\usepackage{pdfpcnotes}

\usepackage[basque]{babel}

\newcommand{\specialcell}[2][l]{%
 \begin{tabular}[#1]{@{}l@{}}#2\end{tabular}}

\usepackage{caption}
\usepackage{eurosym}

\usepackage{listings}

\makeatletter%
\special{pdf: put @thispage <</Group << /S /Transparency /I true /CS /DeviceRGB>> >>}%
\makeatother%

\makeatletter
\let\@@magyar@captionfix\relax
\makeatother

\usepackage{ragged2e}


